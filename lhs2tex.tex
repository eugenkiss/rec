\documentclass[
    a4paper,
    % 10pt,
    % DIV10,
%   twoside,
    oneside,
    parskip=half,
    toc=listof,
    bibliography=totoc,
    listof=totoc,
]{scrbook}

\usepackage[utf8]{inputenc}
\usepackage[ngerman]{babel}

\usepackage{url}
\usepackage[stretch=30]{microtype}
\usepackage{longtable}
\usepackage{tabularx}
\usepackage{graphics}
\usepackage{verbatim}
\usepackage{listings}
\usepackage{array}
\usepackage[refpages]{gloss}
\usepackage{amsmath}
\usepackage{amssymb}
\usepackage{rotating}
\usepackage{color}
\usepackage{comment}

\lstdefinelanguage{Rec}
  {emph={if,then,else},emphstyle=\bfseries,columns=flexible,
  literate={<=}{{$\leq$}}2 {>=}{{$\geq$}}2 {!=}{{$\neq$}}2
  {*}{{$\cdot$}}1 {/}{{$\div$}}1
  }

\lstdefinelanguage{Goto}
  {basicstyle=\ttfamily,
  emph={IF,THEN,ELSE,END,GOTO,PEEK,PUSH,POP,RETURN,CALL},columns=flexible,
  literate={<=}{{$\leq$}}2 {>=}{{$\geq$}}2 {!=}{{$\neq$}}2
  {*}{{$\cdot$}}1 {/}{{$\div$}}1
  }

\newenvironment{myindent}[1]%
 {\begin{list}{}%
         {\setlength{\leftmargin}{#1}}%
         \item[]%
 }
 {\end{list}}

%include lhs2tex.fmt
\begin{document}
\chapter{Quellcode}

Einführung erklüarung...

Fokus der Kommentare liegt hier ofensichtlich mehr auf der konkreten
Implementierung als auf der Hintergrundkonzepte...

Quellcode wurde im Literate Programming Stil verfasst.

GHC Version, cabal file, folder aufbau, tests...

Es wird davon ausgegangen, dass der Leser ein wenig vertraut mit Haskell ist.

Bei der Darstellugn von Quellcode werden einige Freiheiten verwendet, sodass er
ästhetischer ist ...

Die Ordnerstruktur des Projekts sieht folgendermaßen aus\footnote{Unwichtige
Dateien sind aus Gründen der Prägnanz nicht aufgelistet.}:
%
\begin{myindent}{3mm}
\begin{verbatim}
|-- recgoto.cabal
|-- Setup.hs
|-- src
|   |-- Rec.lhs
|   |-- Goto.lhs
|   |-- GotoToRec.lhs
|   |-- Util.lhs
|   `-- cmdtools
|       |-- RecCmdTool.lhs
|       `-- GotoCmdTool.lhs
`-- examples
    |...
\end{verbatim}
\end{myindent}
%
@recgoto.cabal@ und @Setup.hs@ werden für die Installation des Projektes
benötigt. Die Installtion kann mit dem Kommandozeilenbefehl @cabal install@ im
Projektordner geschehen. Dabei werden die Kommandozeilenprogramme @rec@ und
@goto@ installiert mit denen unter anderem entsprechende Programmtexte
ausgeführt werden können. Im Ordner @examples@ befinden sich eingie
Beispielprogrammtexte mit denen die Kommandozeilenprogramme ausprobiert werden
können.  In dem Ordner @src@ befindet
sich der Quellcode dieses Projekts.

Im Folgenden wird zuerst der Quellcode mitsamt Kommentaren der Dateien
@Rec.lhs@ und @RecCmdTool.lhs@ aufgeführt, gefolgt von @Goto.lhs@,
@GotoToRec.lhs@\footnote{Die Aufspaltung in eine @Goto.lhs@ und eine
@GotoToRec.lhs@ Datei hat rein technische Gründe. Und zwar kann der GHC
Compiler nicht mit zyklischen Modulen umgehen auch wenn die Haskell
Spezifikation dieses erlaubt.} und @GotoCmdTool.lhs@. Schliesslich wird noch
@Util.lhs@ angehangen.

%include src/Rec.lhs

%include src/cmdtools/RecCmdTool.lhs

%include src/Goto.lhs

%include src/GotoToRec.lhs

%include src/cmdtools/GotoCmdTool.lhs

% include src/Util.lhs

\end{document}
