\documentclass[
    a4paper,
    % 10pt,
    % DIV10,
%   twoside,
    oneside,
    parskip=half,
    toc=listof,
    bibliography=totoc,
    listof=totoc,
]{scrbook}

\usepackage[utf8]{inputenc}
\usepackage[ngerman]{babel}

\usepackage{url}
\usepackage[stretch=30]{microtype}
\usepackage{longtable}
\usepackage{tabularx}
\usepackage{graphics}
\usepackage{verbatim}
\usepackage{listings}
\usepackage{array}
\usepackage[refpages]{gloss}
\usepackage{amsmath}
\usepackage{amssymb}
\usepackage{rotating}
\usepackage{color}
\usepackage{comment}

\definecolor{LinkColor}{rgb}{0,0,0}
\usepackage{hyperref}
\hypersetup{colorlinks=true,%
    linkcolor=LinkColor,%
    citecolor=LinkColor,%
    filecolor=LinkColor,%
    menucolor=LinkColor,%
    urlcolor=LinkColor}

\lstdefinelanguage{Rec}
  {basicstyle=\sffamily,mathescape,
  emph={if,then,else},emphstyle=\bfseries,columns=flexible,
  literate={<=}{{$\leq$}}2 {>=}{{$\geq$}}2 {!=}{{$\neq$}}2
  {\\}{{$\lambda$}}1
  {*}{{$\cdot \ $}}1 {/}{{$\div$}}2 {//}{{//}}2 {/*}{{/*}}2 {*/}{{*/}}2
  {M1}{{M$_1$}}2
  }

\lstdefinelanguage{Goto}
  {basicstyle=\ttfamily,mathescape,
  emph={IF,THEN,ELSE,END,GOTO,PEEK,PUSH,POP,RETURN,CALL},columns=fixed,
  literate={<=}{{$\leq$}}2 {>=}{{$\geq$}}2 {!=}{{$\neq$}}2
  {*}{{$\cdot \ $}}1 {/}{{$\div$}}1 {^}{{$^\wedge$}}1 {\%}{{$\%$}}1
  {//}{{//}}2 {/*}{{/*}}2 {*/}{{*/}}2
  {M1}{{M$_1$}}2 {M2}{{M$_2$}}2
  {x0}{{x$_0$}}2 {x1}{{x$_1$}}2 {x2}{{x$_2$}}2 {x3}{{x$_3$}}2 {x4}{{x$_4$}}2
  {xi}{{x$_i$}}2 {xj}{{x$_j$}}2 {xk}{{x$_k$}}2
  {t0}{{t$_0$}}2 {t1}{{t$_1$}}2 {t2}{{t$_2$}}2
  }

\newenvironment{myindent}[1]%
 {\begin{list}{}%
         {\setlength{\leftmargin}{#1}}%
         \item[]%
 }
 {\end{list}}

\title{Quellcode zur Bachelor-Arbeit
``Funktionale Programmierung und Berechenbarkeit''}
\author{Evgenij Kiss}

% disables chapter, section and subsection numbering
\setcounter{secnumdepth}{-1}

%include lhs2tex.fmt
\begin{document}
\maketitle
\tableofcontents

\chapter{Einführung}

Diese Datei beinhaltet den Quellcode und die Dokumentation zu der
Implementierung der Compiler für die Sprachen, die ich in meiner Bachelor-Arbeit
``Funktionale Programmierung und Berechenbarkeit'' erstellt habe. Der Quellcode
ist im \emph{Literate Programming}-Stil gehalten. Das bedeutet, grob, dass der
Fokus des Quellcodes eher auf einer nachvollziehbaren Dokumentation liegt und
die Reihenfolge der Programmausdrücke diesbezüglich ausgelegt ist. Dennoch
handelt es sich hier nicht einfach um den Inhalt der Bachelorarbeit erweitert um
Programmtexte. Hat der Text der Bachelorarbeit eher Wert auf die allgemeine
Beschreibung der Ideen gelegt, so liegt der Fokus dieses Textes hauptsächlich
auf den Implementierungsdetails, wobei doch versucht wird immerhin knapp auf die
allgemeinen Ideen an passender Stelle einzugehen. Trotzdem ist es ratsam, sich
zuerst die Bachelor-Arbeit durchzulesen, da ansonsten einige Begründungen wenig
Sinn machen werden. Weiterhin wird davon ausgegangen, dass der Leser vertraut
ist mit Haskell und einigen, in diesem Projekt verwendeten, Bibliotheken
(speziell \emph{Parsec}).

Die Ordnerstruktur des Projekts sieht folgendermaßen aus\footnote{Unwichtige
Dateien sind aus Gründen der Prägnanz nicht aufgelistet.}:
%
\begin{myindent}{3mm}
\begin{verbatim}
|-- recgoto.cabal
|-- lhs2tex.tex
|-- code2pdf.sh
|-- Setup.hs
|-- src
|   |-- Rec.lhs
|   |-- Goto.lhs
|   |-- GotoToRec.lhs
|   |-- Util.lhs
|   `-- cmdtools
|       |-- RecCmdTool.lhs
|       `-- GotoCmdTool.lhs
`-- examples
    |...
\end{verbatim}
\end{myindent}
%
Um das Projekt zu installieren wird eine Haskell-Installation benötigt. Es wird
davon ausgegangen, dass der GHC Haskell-Compiler in der Version 7.4 installiert
ist. @recgoto.cabal@ und @Setup.hs@ werden für die Installation des Projektes
benötigt. Die Installtion kann mit dem Kommandozeilenbefehl @cabal install@ im
Projektordner geschehen. Dabei werden die Kommandozeilenprogramme @rec@ und
@goto@ installiert mit denen unter anderem entsprechende Programmtexte
ausgeführt werden können. Im Ordner @examples@ befinden sich einige
Beispielprogrammtexte mit denen die Kommandozeilenprogramme ausprobiert werden
können. In dem Ordner @src@ befindet sich der Quellcode dieses Projekts. Um
diese PDF zu generieren muss das Skript @code2pdf.sh@ ausgeführt werden. Dazu
muss zusätzlich eine Latex-Distribution auf dem System vorhanden sein und das
Programm \emph{lhs2tex}.

Im Folgenden wird zuerst der Quellcode mitsamt Kommentaren der Dateien
@Rec.lhs@ und @RecCmdTool.lhs@ aufgeführt, gefolgt von @Goto.lhs@,
@GotoToRec.lhs@\footnote{Die Aufspaltung in eine @Goto.lhs@ und eine
@GotoToRec.lhs@ Datei hat rein technische Gründe. Und zwar kann der GHC
Compiler nicht mit zyklischen Modulen umgehen auch wenn die Haskell
Spezifikation dieses erlaubt.} und @GotoCmdTool.lhs@. Schliesslich wird noch
@Util.lhs@ angehangen.

%include src/Rec.lhs

%include src/cmdtools/RecCmdTool.lhs

%include src/Goto.lhs

%include src/GotoToRec.lhs

%include src/cmdtools/GotoCmdTool.lhs

% include src/Util.lhs

\end{document}